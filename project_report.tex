\documentclass[11pt]{amsart}
\usepackage{geometry}                % See geometry.pdf to learn the layout options. There are lots.
\geometry{letterpaper}                   % ... or a4paper or a5paper or ... 
%\geometry{landscape}                % Activate for for rotated page geometry
%\usepackage[parfill]{parskip}    % Activate to begin paragraphs with an empty line rather than an indent
\usepackage{graphicx}
\usepackage{amssymb}
\usepackage{epstopdf}
\usepackage{enumerate}
\usepackage{listings}
\DeclareGraphicsRule{.tif}{png}{.png}{`convert #1 `dirname #1`/`basename #1 .tif`.png}
\setlength{\parindent}{0pt}  

\usepackage{fancyhdr}
\pagestyle{fancy}
\lhead{\footnotesize \parbox{11cm}{CIS 550 Project Report} }
\rhead{\footnotesize \parbox{11cm}{Max Scheiber, Brian Shi, and Rigel Swavely} }

\title{CIS 550 Project Report - Max Scheiber, Brian Shi, and Rigel Swavely}
\date{12.16.2014}

\begin{document}
\maketitle
\section{Introduction}
We built {\bf EIGHT}, a travel planning website and social network. The name was inspired by an
inside joke among CIS 121 TAs and all too much box wine.

\section{Project Goals}
We had three goals for this project. One was to build a scalable application at both the
database level and the server level. We wanted to design a schema that was easy to build more
features on top of, and we also wanted our application to have good performance (i.e. not just
be another class project). Next, we wanted to learn more about Node.js. Max had done some
consulting work in Node, but it was pretty new to both Brian and Rigel. Finally, we had
traditionally used ORMs or ODMs when writing web applications in the past, so we figured
that it would be a different and refreshing experience to have to write raw SQL this time around.

\section{Application Details}
\subsection{Architecture}
The project was build as a Node application on the Express framework, and used the Model-View-Controller 
architectural pattern. We used strict separation of concerns and loose coupling for our models. Like
how other applications will place ORMs and ODMs in their models, we used our models as a convenient
wrapper interface to the database. We used embedded JavaScript (EJS) for our views, and generally tried
to keep them stateless, with the exception of some checks to see if the visitor is logged in. We
definitely used the fat controller paradigm, placing a lot of application logic in our controllers. \\

We availed ourselves of several useful APIs for terseness. We used Underscore.js to write JavaScript
with functional paradigms, the request library for programmatically downloading images into our
cache, and the MongoDB library to give simple bindings to Mongo. \\

The client and server communicate through RESTful HTTP calls. Our website is not extremely dynamic,
but it would have been straightforward to use AJAX or web sockets to achieve this.

\subsection{Implementation}
\subsubsection{Photos and Videos}
Uploading photos and videos was fairly straightforward. The url referring to the photo or video was given by
the user in one of the views, which is then passed into the database to create a new row in its respective table.
A corresponding entry is created in Media, which contains the privacy information, and in Owns, which contains
the relation to the user that uploaded the photo or video.\\

Implementing privacy restrictions was done serverside. When loading photos or videos, if the setting is `private,'
it will only display if the current user is the owner. If the setting is `sharedWithTripMembers,' then it only displays
if the current user is a member of the trip that the media is associated with.
\subsubsection{Recommendations} We used a very simple algorithm to determine recommendations. First, we look at all of the
trips that a user is on, and if any of the trips include destinations that are not part of the user's profile, the destination
is recommended. Similarly, if there are any members of those trips that the user is not already friends with, then they are
recommended as friends.
\subsubsection{Search}
We use a simple search algorithm based on Damerau-Levenshtein distance. This is
obviously not the most intelligent search algorithm - we get some weird results
toward the bottom of the search page - but it generally works for catching typos
and immediatlely relevant results. It would have been better to use some sort of
PageRank algorithm to group together relevant destinations on a trip, as one
possible improvement.
\subsubsection{Caching, NoSQL}
We did our caching in MongoDB, hosted on Amazon in the same region (us-west) as
the rest of our stack. We move popular photos to our cache by tracking their
view count. After being moved to the cache, we obviously read the binary data
from there instead of loading from the URL. We do not currently purge the cache
due to the artificial nature of the project, but we could easily implement a
least recently used (LRU) strategy if we needed to.
\subsubsection{Additional Extra Credit Features}
We implemented searchable hashtags for classifying media on the website - specifically, photos,
trips, and destinations. Unlike standard search, these are evaluated strictly (like Twitter)
and are useful for viewing relationships between disparate sets of user-created here. \\

We also implemented Bing image search integration. When you perform a search on our website,
we also show five relevant images from Bing search. This works extremely well.

\section{Technical Challenges}
Throughout the course of the project, we encountered several technical challenges. The first of such 
challenges was deploying our web application to the cloud. Although all members of our team had 
worked on developing web applications in the past, none of us had used platforms similar to Amazon EC2.
Figuring out how to set up the account properly with the correct security and firewall settings proved
to be a bit of a challenge, but after carefully following the instructions provided through the course
we were able to successfully log in to a server. Finally, once we set up the proper software on the server
such as Node and Git, we were able to start the Node server and access our application.\\

In addition, connecting to Oracle from our web application proved to be fairly difficult. Although there are
commonly used, well maintained Node packages to connect to most database systems like PostgreSQL, MongoDB,
Redis and more, there is not much support for connecting to an Oracle database. Thankfully, we were able to find
one package, but installing it turned out to be fairly difficult. The package required software that had to be
separately downloaded from Oracle in order to work, and configuring it took several tries both on our local machines
and on the server. Once it was finally set up, however, using the interface to interact with the database was
relatively straightforward.\\

Finally, we had some trouble creating a schema for the database that suited our requirements. We wanted to have
one common interface to go between our different media types, such as trips, destinations and photos, and user posted
content regarding those media types, such as ratings and comments. Additionally, we wanted to avoid having multiple 
likes, commments, and ratings tables for each type of media (eg. photo likes, trip likes, etc), and instead
have one common set of tables among all of them. However, that would require a relationship from every media
table to the ratings, likes, and comments tables, which still seemed wasteful. Additionally, if we were to implement
another type of user created content, such as hashtags, it would require changing every table to include a
new relationship under this implementation. Instead, we decided to have one central Media table, which had references to
every entry in each of our specific media tables. The media table was indexed by a primary key which was a composite
of the relevant media's type and its pid in its own table. For example, if we had a photo with primary key 12 in the Photo table, its corresponding primary key in the Media table would be (`Photo', 12). The ratings, likes, and comments tables then simply used the primary key of the entry in the Media table that it referred to, minimizing duplication.

\section{Performance Evaluation}
We used Apachebench to benchmark our application. We hypothesized that the main bottleneck of our
application was the constant calls to Oracle - we assumed we were IO-bound on the server's end,
rather than IO-bound at the client's end (i.e. network latency) or CPU-bound. We therefore conducted
several tests to determine whether our hypothesis was correct.

\subsection{Local versus EC2}
We first attempted to control for running our application on {\tt localhost} versus hosting it on
Amazon. We therefore ran 100 sequential requests to load the profile of username {\tt rigel},
who had a lot of photos and trips. \\

\begin{tabular}{|c|c|c|}
  \hline {\bf Description} & {\bf Mean (sec.)} & {\bf SD (sec.)} \\
  \hline {\tt localhost} & 3593 & 705 \\
  \hline EC2 & 408 & 199 \\
  \hline
\end{tabular} \\

From this, we saw an interesting effect of spatial locality. We would have expected the
{\tt localhost} calls to be quicker. However, the fact that the Oracle database and EC2 instance
are both on us-west, combined with the simple nature of the EC2 server being beefy, made the
cloud calls much quicker. This surprised us, but it makes sense in retrospect.

\subsection{CPU versus IO}
We next attempted to discern whether our latency was CPU-bound or IO-bound. We therefore ran 100 
sequential requests to load the profile of username {\tt max}, who had few photos and trips, and the profile of username {\tt rigel}, who had a lot of photos and trips. \\

\begin{tabular}{|c|c|c|}
  \hline {\bf Description} & {\bf Mean (sec.)} & {\bf SD (sec.)} \\
  \hline {\tt localhost}, {\tt max} & 1017 & 102 \\
  \hline EC2, {\tt max} & 283 & 60.6 \\
  \hline {\tt localhost}, {\tt rigel} & 3593 & 705 \\
  \hline EC2, {\tt rigel} & 408 & 199 \\
  \hline
\end{tabular} \\

From this, we can infer that a lot of latency comes from the Oracle database. Loading all 42 of
{\tt rigel}'s photos added a quarter of a second to the page load time on average. This is also
likely the reason that {\tt rigel} had a higher deviation on page load times.

\subsection{Sequential versus concurrent}
{\em Note}: This section is for extra credit, as delineated by the writeup. \\

Let us explore the time needed to load {\tt rigel}'s profile as a function of concurrent requests. \\

\begin{tabular}{|c|c|c|c|c|c|}
  \hline {\bf Concurrency} & {\bf Min (sec.)} & {\bf Mean (sec.)} & {\bf SD (sec.)} & 
    {\bf Median (sec.)} & {\bf Max (sec.)} \\
  \hline 1 & 311 & 408 & 199 & 338 & 1843 \\
  \hline 2 & 312 & 427 & 201 & 372 & 2075 \\
  \hline 3 & 316 & 500 & 233 & 429 & 2600 \\
  \hline 5 & 327 & 688 & 492 & 578 & 6925 \\
  \hline 10 & 351 & 744 & 313 & 672 & 3586 \\
  \hline 15 & 416 & 1154 & 626 & 968 & 6492 \\
  \hline 25 & 555 & 1773 & 806 & 1601 & 9173 \\
  \hline
\end{tabular} \\

{\em Methodology}: We made requests like {\tt ab -n 1000 -c 10 [url]}, where we kept the number of
total requests ten times the number of concurrent requests. \\

Our application can handle concurrency reasonably well; it handles up to 10 simultaneously heavy
requests with reasonable latency. Node's event-driven nature facilitates this. We conjecture that
Oracle is not able to handle more than ten simultaneously requests gracefully; we do not have
hard evidence of this, but judging by our server's logs, Oracle started clumping together similar
database calls at this point. For example, the logs would have seven or eight consecutive
responses from the {\tt Photo} table, even though we also made subsequent calls to the {\tt Trip}
table (among others). \\

We finally noticed the long tail on these statistics. Our application will every now and then take
a preposterous amount of time to load a page. We aren't positive why this is, but speculate that it
may be because of quirks in the OS scheduler on the server or database. \\

\section{Potential Future Extensions}
While this web application has a mutltitude of features, there is a lot of room to expand. For example, one
feature that is missing from our implementation that is common on most social networks is notifications.
Users could be notified of events that happen that are directly related to them, such as a new
destination or photo being added to a trip they are on. Trip requests and friend requests, which are currently
rendered in different places, could be combined into this stream.\\

Another simple feature to extend the application would be sharable links, which would be similar to 
photos and videos. Users would be able to share links to web pages such as articles, which would then
be likeable, rateable, commentable, and would show up on other users newsfeeds. Additionally, users
could associate the links with trips or with destinations, the same way albums and destinations are 
associated with trips.\\

These are just two basic examples of possible extensions. Of course, the full range of possible extensions is
enormous - we could integrate chat, Yelp, hotel booking websites such as Expedia and Kayak, budget management, 
and so on.



\end{document}
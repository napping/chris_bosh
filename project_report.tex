\documentclass[11pt]{amsart}
\usepackage{geometry}                % See geometry.pdf to learn the layout options. There are lots.
\geometry{letterpaper}                   % ... or a4paper or a5paper or ... 
%\geometry{landscape}                % Activate for for rotated page geometry
%\usepackage[parfill]{parskip}    % Activate to begin paragraphs with an empty line rather than an indent
\usepackage{graphicx}
\usepackage{amssymb}
\usepackage{epstopdf}
\usepackage{enumerate}
\usepackage{listings}
\DeclareGraphicsRule{.tif}{png}{.png}{`convert #1 `dirname #1`/`basename #1 .tif`.png}
\setlength{\parindent}{0pt}  

\usepackage{fancyhdr}
\pagestyle{fancy}
\lhead{\footnotesize \parbox{11cm}{CIS 550 Project Report} }
\rhead{\footnotesize \parbox{11cm}{Max Scheiber, Brian Shi, and Rigel Swavely} }

\title{CIS 550 Project Report - Max Scheiber, Brian Shi, and Rigel Swavely}
\date{12.16.2014}

\begin{document}
\maketitle
\section{Technical Challenges}
Throughout the course of the project, we encountered several technical challenges. The first of such 
challenges was deploying our web application to the cloud. Although all members of our team had 
worked on developing web applications in the past, none of us had used platforms similar to Amazon EC2.
Figuring out how to set up the account properly with the correct security and firewall settings proved
to be a bit of a challenge, but after carefully following the instructions provided through the course
we were able to successfully log in to a server. Finally, once we set up the proper software on the server
such as Node and Git, we were able to start the Node server and access our application.\\

In addition, connecting to Oracle from our web application proved to be fairly difficult. Although there are
commonly used, well maintained Node packages to connect to most database systems like PostgreSQL, MongoDB,
Redis and more, there is not much support for connecting to an Oracle database. Thankfully, we were able to find
one package, but installing it turned out to be fairly difficult. The package required software that had to be
separately downloaded from Oracle in order to work, and configuring it took several tries both on our local machines
and on the server. Once it was finally set up, however, using the interface to interact with the database was
relatively straightforward.\\

Finally, we had some trouble creating a schema for the database that suited our requirements. We wanted to have
one common interface to go between our different media types, such as trips, destinations and photos, and user posted
content regarding those media types, such as ratings and comments. Additionally, we wanted to avoid having multiple 
likes, commments, and ratings tables for each type of media (eg. photo likes, trip likes, etc), and instead
have one common set of tables among all of them. However, that would require a relationship from every media
table to the ratings, likes, and comments tables, which still seemed wasteful. Additionally, if we were to implement
another type of user created content, such as hashtags, it would require changing every table to include a
new relationship under this implementation. Instead, we decided to have one central Media table, which had references to
every entry in each of our specific media tables. The media table was indexed by a primary key which was a composite
of the relevant media's type and its pid in its own table. For example, if we had a photo with primary key 12 in the Photo table, its corresponding primary key in the Media table would be (`Photo', 12). The ratings, likes, and comments tables then simply used the primary key of the entry in the Media table that it referred to, minimizing duplication.

\end{document}